%%%%%%%%%%%%%%%%%%%%%%%%%%%%%%%%%%%%%%%%%
% Curriculum Nicolás Friz Pereira
% Adaptado de plantilla DeveloperCV con IBM Plex Sans
%%%%%%%%%%%%%%%%%%%%%%%%%%%%%%%%%%%%%%%%%

\documentclass[9pt]{developercv}

% Usamos XeLaTeX/LuaLaTeX: fontspec para fuentes modernas
\usepackage{fontspec}

% Definimos IBM Plex Sans como fuente principal (se instala en el workflow)
\newfontfamily\ibmplex{IBM Plex Sans}[
  UprightFont    = *-Regular,
  BoldFont       = *-Bold,
  ItalicFont     = *-Italic,
  BoldItalicFont = *-BoldItalic,
  Scale          = MatchLowercase,
]

% Hacemos que la familia por defecto sea la fuente sans definida
\renewcommand\familydefault{\sfdefault}
\setmainfont{IBM Plex Sans}

% Codificación y colores
\usepackage[T1]{fontenc}
\usepackage{xcolor}
\definecolor{headerbg}{HTML}{2E5F5A}
\definecolor{headerfg}{HTML}{FFFFFF}

\pagestyle{empty}

\begin{document}

%----------------------------------------
% Encabezado Centrado
%----------------------------------------
\begin{center}
  \colorbox{headerbg}{%
    \parbox{0.95\linewidth}{%
      \vspace{1em}
      \centering
      % Nombre en IBM Plex Sans Bold
      {\ibmplex\Huge\color{headerfg}\textbf{Nicolás Friz Pereira}}\\[4pt]
      {\ibmplex\large\color{headerfg}\textbf{Ingeniero Informático}}\\[1pt]
      {\small\color{headerfg}
        \href{mailto:ni.frizp@gmail.com}{ni.frizp@gmail.com} \quad $|$ \quad
        Cauquenes, Maule, Chile \quad $|$ \quad
        \href{tel:+56933134466}{+56 9 3313 4466}
      }
      \vspace{1em}
    }
  }
  \vspace{0.5cm}
\end{center}

% Contact Icons
\begin{center}
  \begin{tabular}{c@{\hspace{1.5cm}}c@{\hspace{0.5cm}}c}
    \icon{Linkedin}{12} {\href{https://www.linkedin.com/in/nicolasfrizpereira}{/in/nicolasfrizpereira}} &
    \icon{Github}{12}{\href{https://github.com/niFrizP}{github.com/niFrizP}} &
    \icon{Language}{12}{Inglés: A2 (TOEIC 480)}
  \end{tabular}
\end{center}

%----------------------------------------
% Profile
%----------------------------------------
\cvsect{Sobre mi}

Mi nombre es \textbf{Nicolás Friz Pereira}, soy Ingeniero Informático, cuento con experiencia en desarrollo de aplicaciones web, aseguramiento de la calidad de software (QA) y optimización de procesos operativos. Me caracterizan el aprendizaje autodidacta, la responsabilidad y el enfoque en la mejora continua, así como la disposición para asumir nuevos retos y adquirir conocimientos a partir de ellos, a su  con experiencia en armados y mantenimientos de computadoras,

%----------------------------------------
% Experiencia
%----------------------------------------
\cvsect{Experiencia}
\begin{itemize}
  \item \textbf{febrero 2025 -- octubre 2025 | Desarrollador web y Asesoría en TI – | TEQMED SpA, Concepción.}
  \\Desarrollo web (Laravel, PHP, JavaScript, TailwindCSS y MySQL). Desarrollo de Intranet y sistema de Tickets para servicio técnico y asesorías en informática.

  \item \textbf{agosto 2024 -- febrero 2025 | Analista QA (Programa Blue Journey) – | IBM Chile, Santiago.}
  \\Ejecución de pruebas funcionales y de regresión bajo metodología Scrum. Reporte y documentación de errores. Definición de criterios de aceptación y métricas de calidad.

  \item \textbf{agosto 2022 -- agosto 2024 | Desarrollador FullStack y Especialista TI – | TEQMED SpA, Concepción.}
  \\Desarrollo y mantenimiento de sistemas internos (PHP, MySQL y Bulma CSS). Generación de reportes y dashboards. Gestión de infraestructura tecnológica y soporte técnico.

  \item \textbf{abril 2022 -- agosto 2024 | Líder de Taller de Impresión 3D – | CITT DuocUC, Concepción.}
  \\Planificación y realización de talleres para estudiantes en diseño e impresión 3D.

  \item \textbf{diciembre 2019 | Asistente de Apoyo – | ABCDIN, Cauquenes.}
  \\Apoyo logístico y atención al cliente durante campaña navideña año 2019.
\end{itemize}

%----------------------------------------
% Educación
%----------------------------------------
\cvsect{Educación}
\begin{itemize}
  \item \textbf{2020 -- 2025 | Ingeniería en Informática.}
  \\DuocUC, Concepción, Región del Biobío.

  \item \textbf{2018 -- 2019 | Técnico en Administración Computacional especializado en Armado de Computadoras. }
  \\Instituto Conosur, Cauquenes, Región del Maule.

    \item \textbf{2008 -- 2019 | Enseñanza Basica y Media Humanista cientifica. }
  \\Liceo Antonio Varas, Cauquenes, Región del Maule.
\end{itemize}

%----------------------------------------
% Habilidades y Certificaciones
%----------------------------------------
\cvsect{Habilidades y Certificaciones}
\begin{minipage}[t]{0.5\textwidth}
\vspace{3pt}
\textbf{Habilidades Técnicas}
\vspace{3pt}
\begin{itemize}
  \item \textbf {Lenguajes:} PHP, JavaScript, Python y LaTeX.
  \item \textbf {Frameworks:} Laravel, Angular, Ionic y NextJS.
  \item \textbf {CMS:} WordPress.
  \item \textbf {Bases de Datos:} MySQL, SQL Server y Oracle PL/SQL.
  \item \textbf {Frontend:} HTML, CSS, Sass, Tailwind CSS y Bulma CSS.
  \item \textbf {Multimedia:} Photoshop, Illustrator, Lightroom, Sony Vegas Pro, Premiere y DaVinci Resolve.
  \item \textbf {Hardware y TI:} Ensamblaje de PC, Impresión 3D y Soporte técnico.
\end{itemize}
\end{minipage}
\begin{minipage}[t]{0.5\textwidth}
\vspace{3pt}
\textbf{Certificaciones y Logros}
\begin{itemize}
  \item \textbf{2025:} IBM Z Day SE - Seguridad.
  \item \textbf{2024:} IBM Agile Explorer, Quantum Conversations yㅤ 
  ㅤWatsonX Essencials
  \item \textbf{2023:} Diseño e Impresión 3D - Full3D \& Creality.
  \item \textbf{2022:} {\href{https://www.credly.com/badges/06d78a0f-69f4-4093-b0bd-3b9f3b5f69cc/}{Principios de Diseño Digital en Adobe Photoshop  - Universidad Tecmilenio (México)}}
  \item \textbf{2022:} 1er Lugar, Innova Challenge - Universidad Tecmilenio (México). 
\end{itemize}
\end{minipage}

\end{document}